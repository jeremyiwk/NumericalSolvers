\documentclass[10pt]{article}
\usepackage{geometry}                % See geometry.pdf to learn the layout options. There are lots.
\geometry{letterpaper}                   % ... or a4paper or a5paper or ... 
%\geometry{landscape}                % Activate for for rotated page geometry
\usepackage[parfill]{parskip}    % Activate to begin paragraphs with an empty line rather than an indent

%%%%%%%%%%%%%%%%%%%%
\newcommand{\hide}[1]{}

\usepackage{natbib}
\usepackage{xcolor}
\usepackage{url}
\usepackage{hyperref}
\usepackage{mathtools}
\usepackage{units}

\hide{
\usepackage{amscd}
\usepackage{amsfonts}
\usepackage{amsmath}
\usepackage{amssymb}
\usepackage{amsthm}
\usepackage{cases}		 
\usepackage{cutwin}
\usepackage{enumerate}
\usepackage{epstopdf}
\usepackage{graphicx}
\usepackage{ifthen}
\usepackage{lipsum}
\usepackage{mathrsfs}	
\usepackage{multimedia}
\usepackage{wrapfig}
\usepackage{gensymb}
\usepackage{mathop}
\usepackage{mathbb}
\usepackage{mathcal}
\usepackage{amssymb}
\usepackage{listings}
%\usepackage{pythontex}

\allowdisplaybreaks
}

%\bibliographystyle{humanbio}

	 
%\input{/usr/local/LATEX/Lee_newcommands.tex}
\newcommand{\itemlist}[1]{\begin{itemize}#1\end{itemize}}
\newcommand{\enumlist}[1]{\begin{enumerate}#1\end{enumerate}}
\newcommand{\desclist}[1]{\begin{description}#1\end{description}}

\newcommand{\Answer}[1]{\begin{quote}{\color{blue}#1}\end{quote}}
\newcommand{\AND}{\wedge}
\newcommand{\OR}{\vee}
\newcommand{\ra}{\rightarrow}
\newcommand{\lra}{\leftrightarrow}
%\DeclareMathOperator*{\E}{\mathbb{E}}

\title {Using Github}
\author{Matt Kafker}


\usepackage{listings}

\begin{document}
\maketitle

\subsection*{Getting Started}
Open a Terminal window, and navigate to a place in your file system where you want all our code to live.

Then, in the command line, enter


\begin{verbatim}
	git clone https://github.com/mmkafker/NumericalSolvers.git
\end{verbatim}

(You might have to figure out how to configure github on your computer. Good luck.) Once you have done this, there should be a folder in your file system entitled ``NumericalSolvers.''

\subsection*{Daily Usage}
Every time you want to work on the project, navigate to the ``NumericalSolvers'' folder and type 

\begin{verbatim}
	git pull
\end{verbatim}
You will probably be asked to log in blah blah. Once you have done this, your code will be up to date with the repository.    

After you have made edits to the code, assuming you want those edits to be permanently reflected in the shared repository, you do

\begin{verbatim}
	git add *
\end{verbatim}
(or \verb|git add| (files you want to upload)). Then,
\begin{verbatim}
	git commit
\end{verbatim}
which will open a vim window. Try to provide a decent 1-sentence description of the changes you are making to the repository. Then use \verb|:x| to save and close the vim window.

Finally, execute
\begin{verbatim}
	git push
\end{verbatim}
You will have to log in again. Assuming nothing crazy has happened (i.e., we weren't simultaneously editing), then your changes should now be reflected in the repository.

If you want to be super sure, you can do 
\begin{verbatim}
	git pull
\end{verbatim}
again at the very end, so now your local repository is exactly synced with the global one, but this should be redundant if you haven't messed anything up.

    
    
\end{document}  
%%%%%%%%%%%%%%%%%%%%%%%%%%%%%%%%%%%%%%%%%%%%%%